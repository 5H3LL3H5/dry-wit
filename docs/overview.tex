% ====================================================================
%    ___ ___ ___  __ ___ 
%   / _ (_-</ _ \/ _/ _ \
%   \___/__/\___/\__\___/
%
%   ~ La Empresa de los Programadores Profesionales ~
%
%
%  | http://www.osoco.es
%  |
%  | Edificio Moma Lofts
%  | Planta 3, Loft 18
%  | Ctra. Mostoles-Villaviciosa, Km 0,2
%  | Mostoles, Madrid 28935 España
%
% ====================================================================
%
% Copyleft 2015 
%
% by Jose San Leandro
%
%%%%%%%%%%%%%%%%%%%%%%%%%%%%%%%%%%%%%%%%%%%%%%%%%%%%%%%%%%%%%%%%%%%%%%
\documentclass{beamer}

\usetheme{osoco2012}

\usepackage[english]{babel}
\usepackage[utf8]{inputenc}
\usepackage{minted}
\usepackage{listings}

% Metadata
%%%%%%%%%%%%%%%%%%%%%%%%%%%%%%%%%%%%%%%%%%%%%%%%%%%%%%%%%%%%%%%%%%%%%%

\title{dry-wit}
\subtitle{Bash framework}
\author{Jose San Leandro \href{http://twitter.com/rydnr}{@rydnr}}
\institute[OSOCO]{%
  \href{http://www.osoco.es}{OSOCO}
}
\date[10/2015]{October 2015}
\subject{dry-wit}
\keywords{dry-wit, bash, framework}

% Contents
%%%%%%%%%%%%%%%%%%%%%%%%%%%%%%%%%%%%%%%%%%%%%%%%%%%%%%%%%%%%%%%%%%%%%%

\begin{document}

\begin{frame}[plain]
  \titlepage
\end{frame}

\begin{frame}[plain]{Contents}
  \tableofcontents[hideallsubsections]
\end{frame}

\section{Introduction}

\subsection{Motivation}

\begin{frame}{Bash}
	\begin{columns}[t]
		\column{.5\textwidth}
		\begin{block}{In theory}
			\begin{itemize}
				\item Powerful language.
				\item Conventions improve reusability and maintainability.
				\item Great fit for Unix and OSs with strong command-line focus.
			\end{itemize}
		\end{block}
		\pause

		\column{.5\textwidth}
		\begin{block}{In practice}
			\begin{itemize}
				\item Easy for newcomers.
				\item Difficult to master.
                                \item Write once, use twice, dispose.
                                \item Expensive maintenance.
                                \item Hardly reusable.
			\end{itemize}
		\end{block}

	\end{columns}
\end{frame}

\section{dry-wit approach}

\begin{frame}{dry-wit: Features}
  \begin{block}{Features}
    \begin{enumerate}
    \item Provides a consistent way to structure scripts.
    \item Manages required dependencies.
    \item Declarative approach for dealing with errors (Output message + error code).
    \item Handles parameters, flags, and environment variables.
    \item API for creating temporary files, logging, etc.
    \end{enumerate}
  \end{block}
\end{frame}

\section{dry-wit hooks}

\begin{frame}[fragile]{Usage / Help}
  \begin{block}{usage()}
    \inputminted{bash}{usage.bash}
  \end{block}
\end{frame}

\begin{frame}[fragile]{Input: checking}
  \begin{block}{checkInput()}
    \inputminted{bash}{checkinput.bash}
  \end{block}
\end{frame}

\begin{frame}[fragile]{Input: parsing}
  \begin{block}{parseInput()}
    \inputminted{bash}{parseinput.bash}
  \end{block}
\end{frame}

\begin{frame}[fragile]{Script requirements}
  \begin{block}{Dependencies}
    \inputminted{bash}{checkrequirements.bash}
  \end{block}
  \begin{block}{DSL}
    \begin{enumerate}
    \item \textit{executable-file}: The required dependency.
    \item \texttt{message}: The name of a constant describing the error to display should the dependency is not present.
    \item dry-wit checks each dependency and exits if the check fails.
    \end{enumerate}
  \end{block}
\end{frame}

\begin{frame}[fragile]{Error messages}
  \begin{block}{defineErrors()}
    \inputminted[breaklines,fontsize=\tiny]{bash}{defineerrors.bash}
  \end{block}
  \begin{block}{dry-wit takes care of the exit codes}
    \inputminted{bash}{exitwitherrorcode.bash}
  \end{block}
\end{frame}

\begin{frame}[fragile]{main()}
  \begin{block}{main()}
    \inputminted{bash}{main.bash}
  \end{block}
\end{frame}

\section{dry-wit API}

\begin{frame}[fragile]{Logging}
  \begin{block}{logging}
    \inputminted{bash}{logging.bash}
  \end{block}
  \begin{block}{output}
    \inputminted[breaklines,fontsize=\tiny]{bash}{logging-output.bash}
  \end{block}
\end{frame}

\begin{frame}[fragile]{Temporary files}
  \begin{block}{API functions}
    \begin{enumerate}
    \item Functions to create temporary files or folders.
    \item dry-wit takes cares of cleaning them up afterwards.
    \end{enumerate}
  \end{block}
  \begin{block}{createTempFile()}
    \inputminted{bash}{createtempfile.bash}
  \end{block}
  \begin{block}{createTempFolder()}
    \inputminted{bash}{createtempfolder.bash}
  \end{block}
\end{frame}

\begin{frame}[fragile]{Environment variables: Declaration (1/3)}
  \begin{block}{DSL for declaring environment variables}
    \begin{enumerate}
    \item Declared in \textbf{[script].inc.sh}.
    \item Safe to add to version control systems.
    \item Mandatory information: description and default value.
    \end{enumerate}
  \end{block}
  \begin{block}{defineEnvVar()}
    \inputminted{bash}{defineenvvar.bash}
  \end{block}
\end{frame}

\begin{frame}[fragile]{Environment variables: Overridding (2/3)}
  \begin{block}{DSL for overridding default values}
    \begin{enumerate}
    \item Declared in \textbf{.[script].inc.sh}.
    \item Not always safe to add to version control systems.
    \end{enumerate}
  \end{block}
  \begin{block}{overrideEnvVar()}
    \inputminted{bash}{overrideenvvar.bash}
  \end{block}
\end{frame}

\begin{frame}[fragile]{Environment variables: Overridding (3/3)}
  \begin{block}{Environment value overridden from the command line}
    \inputminted{bash}{specifyenvvarcommandline.bash}
  \end{block}
\end{frame}

\end{document}
